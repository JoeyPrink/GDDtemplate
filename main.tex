\documentclass[a4paper]{scrreprt}

%% Language and font encodings
\usepackage[english]{babel}
\usepackage[utf8x]{inputenc}
\usepackage[T1]{fontenc}

%% Sets page size and margins
\usepackage[a4paper,top=3cm,bottom=2cm,left=3cm,right=3cm,marginparwidth=1.75cm]{geometry}

%% Useful packages
\usepackage{amsmath}
\usepackage{graphicx}
\usepackage[colorinlistoftodos]{todonotes}
\usepackage[colorlinks=true, allcolors=blue]{hyperref}

\title{Name of Game}
\subtitle{"YOUR GAME IN ONE LINE" (Witcher e.g.: "Skyrim, but with Story of Game of Thrones")}
\author{AUTHOR}
\titlehead{\centering\includegraphics[width=6cm]{test}}


\begin{document}
\maketitle

\null\vfill
\noindent
Game Design Document Template\\ 
Version v1.1, Nov 2016\\
Version v1.2, Dec 2017\\
Version v1.3, Nov 2019\\
Version v1.4, Mar 2020\\
Copyright 2017-2020 - Johanna Pirker \#tugamedev\\
\newpage

\begin{abstract}
Short abstract of the game (max 150 words) 
\end{abstract}

\tableofcontents
\chapter{Overview}

Main features and aspects of your game on a first page, describing story elements. -> "selling page", publisher should be able to decide after reading this single page whether to buy in or not 

\section{Main Concept}
describe you main concept in one paragraph

\section{Unique Selling Point}
describe you unique selling point in one paragraph


\chapter{References} 
research on similar games, what are the core features, how does your game differ?

\chapter{Specification}
description of target group, platform, art style, who to attract of how to attract 

\section{Player(s) / Target-group}
who is the target group? 

\section{Genre}
what is the genre of the game?

\section{Art Style}
the art style of the game?

\begin{figure}
\centering
\includegraphics[width=0.3\textwidth]{test.jpg}
\caption{\label{fig:art} Art example}
\end{figure}

\section{Forms of Engagement}
thinking of Hunicke's 8 kinds of "fun" - what would you like to focus on?\\
(1. Sensation - Game as sense-pleasure 
2. Fantasy - Game as make-believe
3. Narrative - Game as drama
4. Challenge - Game as obstacle course
5. Fellowship -  Game as social framework
6. Discovery - Game as uncharted territory 
7. Expression - Game as self-discovery 
8. Submission - Game as pastime)

\chapter{Gameplay and Game Setting}
be specific about the core game features 

\section{Mood and Emotions}
what mood and emotions does the game create (can change e.g. for every level / section) 

\section{Story}
the story of the game

\section{World/Environment}
what is the settings of the game 

also, add here a map of your environment or a picture of your world if necessary

\section{Objects in the Game}
what objects will be in the game?

\section{Characters in the Game}
who are the characters in the game?

\section{Main Objective}
what is the goal / main objective of the game?

\section{Core Mechanics}
very important section: what are the core mechanics? be specific

\section{Controls}
describe the controls of the game 
also, add here a controller diagram if necessary 

\chapter{Front End}
description of front end such as start screen, menu screens,..  

\section{Start Screen}

\section{Menus}

\section{End Screen}

\chapter{Technology}
what technologies is the game designed for, what is the target platform, what technologies are used for the development? 

\section{Target Systems}
what platforms is the game designed for

\section{Hardware}
what hardware is needed to play the game? any additional interface? recommended controllers? 

\section{Development Systems/Tools}
please describe the tools you are using (game engine, art tools, ..) 

\chapter{Topic and Inclusion }

describe here how you plan to address the main topic (main theme) and topics around inclusion

\section{Main Theme}
\section{Inclusion}

\subsection{Diversity}
diversity in games is an important topic. please describe here how you addresses diversity in your game and game design elements 
\subsection{Accessibility}
make your games more accessible. use this section to describe what guidelines you addresses and how you cater for  gamers with disabilities and other impairments. great reference: \url{http://gameaccessibilityguidelines.com/}
%\subsection{Humanity}

\chapter{Marketing and Publishing Strategy}

describe here your plan how to get attention for your game (e.g. send to youtubers, twitter strategy, events) 


\chapter{Timeline and Cost Estimation}

In this chapter, you should describe your planned time management, the estimation of how long you think your team will need and how much you think this project would cost. 

Tools we recommend for project management are for instance \url{https://app.hacknplan.com/login}. To track your time, we recommend \url{https://toggl.com/}.  

\begin{table}[h]
\centering
\begin{tabular}{|l|l|l|}
\hline
Milestone & Description & Date \\\hline
& Official Start Date & 01.12.... \\
1 & Milestone Description ..  & 01.12.... \\
2 & Milestone Description ..  & 01.01.... \\
3 & Milestone Description ..  & 01.03.... \\
& End of Project & 01.04.... \\
\hline
\end{tabular}
\caption{\label{tab:schedule}Example Schedule.}
\end{table}

\section{Time Estimation}

While working on your project you should track your time. 
We recommend using \url{https://toggl.com/} for time management. In the final report you will have to compare the estimated time with the actual time. (Miscalculation do not have any effect on your grade!!!)

\section{Cost Estimation}

Estimated cost of the project based the described tasks and milestones and the  time estimation.  

\chapter{Team and Credits}

most important - who are you, who takes what role? 

e.g. :
Project Management: \\
Programming: \\ 
Art: \\ 
Design: \\ 

Additional Credits (e.g. sources of art, audio,.. ) 



%\todo[inline, color=green!40]{This is an inline comment.}

%\bibliographystyle{alpha}
%\bibliography{sample}

\end{document}